\documentclass[a4paper,11pt]{article}
\usepackage[T1]{fontenc}
\usepackage[utf8]{inputenc}
\usepackage{lmodern}
\usepackage{url}
\usepackage{hyperref}
\usepackage{amsmath}
\usepackage{amssymb}
\usepackage{xcolor}
\usepackage{graphicx}
\usepackage{wrapfig}
\usepackage{caption}
\usepackage{multicol}

\hypersetup{colorlinks=true,
  linkcolor=brown,
  urlcolor=blue,
  citecolor=red,
}

\author{Leo Unbekandt \\
 \texttt{leo@unbekandt.eu} \vspace{1em} \\
 \textsc{Cranfield University} \\
 School of engineering \\
 Msc Computational and Software Techniques in Engineering
}
\title{Literature Review}
\date{\today}


\begin{document}

\maketitle
\tableofcontents

\section*{Introduction}

These 10 last years, the Internet has grown at an exponential rate. More and
more people are connected, and using always more numerous kinds of device.
As the demand is raising strongly, service providers have to be able to support
an increasingly amount of customers on their infrastructure. In 2006, Amazon
launched Amazon Web Services (AWS) is born, popularising the concept of cloud
computing. This platform allows developers and companies to allocate resources
on demand, and these are provisionned instantly.

\section{Background}

The market of cloud computing has grown really quickly and forcasts predict
that this expansion won't stop in the near future. The following study gathers
information from different suveys\cite{website:cloudcomputingmarket}.
Currently, the public cloud providers represent a market of 47.4 billion of
dollars and is expecting to grow up to 107.2 billion of dollars on only 4
years.

Beside public clouds, as is defined in the definition of cloud
computing\cite{nistcloudcomputing} by the National Institute of Standards and
Technology, private cloud or hybrid cloud mixing private and public cloud
infrastructure are being developped more and more.  Thanks to open-source
projects like Openstack\cite{website:openstack}, clouds environment can be
installed on private infrastructures. This is necessary, for security,
performance or data control purposes.

It is no coincidence if the industry and the academic environments are really
interested by this domain. On the one hand, it is a field where money is
present and on the second hand, we can consider that we are only at the
beginning of an era: a lot of things have to be developed, to be optimized and
to be found.

The evolution of the paradigm of cloud computing has been made possible thanks
to different technologies. The virtualisation, as explained by Paul Barham et
al.\cite{virtualisation} allows servers to be splitted in different
sub-components, isolated from each other, sharing the resources of the physical
machine. 

Technologies have been developed to give people much more flexibility
in the way to manage their virtual machines, also called instances. Actually,
the concept of live migration, which is detailled in the work of Christopher
Clark et al.\cite{livemigration}, has been built to move instances from one
physical host to another without interrupting the activity of anything running
in the virtual machine. The memory is kept intact of course, but also the
running connections.  The instance may seem frozen for a few second when the
migration is finalized, but nothing is disrupted.

\section{Motivation}

The legitimate question is ``Why do people migrate their infrastructure to a
cloud infrastructure?''. The answers are multiple, Valentina Salapura explains
how a virtualized environment improves the resiliency of an
infrastructure \cite{virtresiliency}. More precisely, when a service requires
to be scalable, highly available and fault taulerant, using cloud technologies
is essential. In the case of disaster recovery scenarios, they are highly
simplified and cheaper thanks to those environments.

As a result the infrastructures are composed of a certain amount of physical
machines (PMs) which could be dispatched among different data centers, and each
of these PMs, containers a variable number of virtual machines (VMs). The
problematic which is now interesting concerns the assignment of these VMs, what
is the optimal distributions of the instances among the different servers? It
depends of what caracteristic has to be optimized.

Thomas Setzer and Alexander Stage base their study on the statement that energy
represents up to 50\% of operating costs of an infrastructure. That's why there
is a need to optimize it. Using the virtual machine reassignment through live
migrations, they are looking at consolidating the VMs on the physical servers.
Consolidating an infrastructure consists in reducing the number of PMs which
are hosting instances without disturbing the performance of these.  After this
operation, useless PMs can be suspended and electricity is saved, then when
more computational power is required they are resumed dynamically.

In the publication of Weihia Song et al.\cite{reassignment:binpacking1} also
introduce their subject by explaining that it has been estimated that Amazon
manages more than half a million of physical servers around the world and that
it must be a priority for them to reduce their expensives by consolidating their
infrastructure.

\section{Algorithms}

\subsection{Linear Programming}

\subsection{Bin packing}

\subsection{Network flows}

\subsection{Others}

\section{Real data analysis}

\section*{Conclusion}

\nocite{*}
\bibliographystyle{unsrt}
\bibliography{literature_review}

\end{document}
